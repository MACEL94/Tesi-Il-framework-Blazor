\chapter{Introduzione}\label{cap:introduzione}

\section{Contesto}\label{sez:contesto}

A seguito della crescita esponenziale del web in questo secolo e dell'abituarsi di coloro che ne usufruiscono ad un livello grafico sempre migliore e ad una esperienza mano a mano pi\`u interattiva e vicina all'utente medio, i siti web e le tecnologie utilizzate si sono adattati per permettere uno sviluppo sempre pi\`u rapido di codice pi\`u facilmente testabile e mantenibile.

Di conseguenza nel frontend si sono susseguiti una serie di framework a partire da JQuery nel 2006, che per primo si \`e occupato di risolvere il problema della compatibilit\`a tra browsers, permettendo ai developers di scrivere una volta, e poter eseguire su tutti i browsers.

AngularJS nel 2010 \`e stato il primo MVC framework ad offrire in un unico pacchetto, two-way data binding, dependency injection, routing facilitato e altri strumenti utili per rendere pi\`u standard lo sviluppo nel frontend~\cite{Hoff}.
Dopo la riscrittura di questo framework nel 2013 che \`e diventato Angular 2 (e recentemente Angular) senza mantenere retrocompatibilit\`a e senza offrire un modo preciso per migrare alla nuova versione agli utilizzatori di AngularJS, React, un nuovo framework pi\`u leggero e modulare sviluppato dagli sviluppatori di Facebook, ha preso il posto di Angular come framework pi\`u utilizzato.

Vue infine \'e il terzo dei principali framework che ha provato a prendere piede proponendo una versione intermedia tra il fortemente opinionato Angular e il pi\'u flessibile React.

Oltre a questi, ciascuno con la propria semantica, organizzazione logica dei folder, spesso una CLI dedicata, ad un developer frontend viene solitamente richiesto di conoscere HTML, CSS e Javascript su cui si basano poi i vari framework.

Oltre a Javascript, se si vuole scrivere degli unit test facilmente mantenibili, bisogna conoscere TypeScript(specialmente se si utilizza Angular, che rende il suo utilizzo obbligatorio) e degli altri framework che facilitino i test(Enzyme, Karma + Jasmine, ...).

\section{Problema}\label{sez:problema}
I continui cambiamenti nei molti framework utilizzati, la diversit\`a degli strumenti stessi tra loco, che spesso realizzano in modo diverso tutti la stessa cosa(framework concorrenti), rendono specialmente per un junior developer molto ampia la curva di apprendimento e lo studio necessario, per essere anche solo operativo.

Oltretutto un developer ad oggi finisce per essere costretto a scegliere se diventare uno sviluppatore frontend o backend, dato che rimanere al passo e aggiornarsi gi\`a in uno di questi due campi richiede tempo e non \`e scontato che venga concesso di poterlo fare in orario lavorativo, pur essendo fondamentale.

Microsoft ha reso nel tempo il framework .NET e le sue implementazioni(.NET Core, .NET Framework e Mono) utilizzabili nei vari linguaggi supportati, C\#, F\# e VB.
Scrivendo ad esempio in C\# \`e possibile sviluppare vari tipi di applicazioni, ma se si decide di sviluppare codice per un applicazione client web ad oggi si \`e ancora costretti a scrivere utilizzando Javascript e un suo framework se si vuole essere utili e competitivi a livello enterprise.

Razor ha provato a risolvere parzialmente il problema di generare codice HTML e CSS in modo dinamico utilizzando C\#, ma \`e utilizzabile solo lato server e quindi ad esempio la cattura di un evento client side come il click di un utente su un bottone senza contattare il server non \`e gestibile utilizzando il solo C\#.